% Options for packages loaded elsewhere
\PassOptionsToPackage{unicode}{hyperref}
\PassOptionsToPackage{hyphens}{url}
%
\documentclass[
  10pt,
  ignorenonframetext,
]{beamer}
\usepackage{pgfpages}
\setbeamertemplate{caption}[numbered]
\setbeamertemplate{caption label separator}{: }
\setbeamercolor{caption name}{fg=normal text.fg}
\beamertemplatenavigationsymbolsempty
% Prevent slide breaks in the middle of a paragraph
\widowpenalties 1 10000
\raggedbottom
\setbeamertemplate{part page}{
  \centering
  \begin{beamercolorbox}[sep=16pt,center]{part title}
    \usebeamerfont{part title}\insertpart\par
  \end{beamercolorbox}
}
\setbeamertemplate{section page}{
  \centering
  \begin{beamercolorbox}[sep=12pt,center]{part title}
    \usebeamerfont{section title}\insertsection\par
  \end{beamercolorbox}
}
\setbeamertemplate{subsection page}{
  \centering
  \begin{beamercolorbox}[sep=8pt,center]{part title}
    \usebeamerfont{subsection title}\insertsubsection\par
  \end{beamercolorbox}
}
\AtBeginPart{
  \frame{\partpage}
}
\AtBeginSection{
  \ifbibliography
  \else
    \frame{\sectionpage}
  \fi
}
\AtBeginSubsection{
  \frame{\subsectionpage}
}
\usepackage{amsmath,amssymb}
\usepackage{lmodern}
\usepackage{iftex}
\ifPDFTeX
  \usepackage[T1]{fontenc}
  \usepackage[utf8]{inputenc}
  \usepackage{textcomp} % provide euro and other symbols
\else % if luatex or xetex
  \usepackage{unicode-math}
  \defaultfontfeatures{Scale=MatchLowercase}
  \defaultfontfeatures[\rmfamily]{Ligatures=TeX,Scale=1}
\fi
\usetheme[]{AnnArbor}
\usecolortheme{dolphin}
\usefonttheme{structurebold}
% Use upquote if available, for straight quotes in verbatim environments
\IfFileExists{upquote.sty}{\usepackage{upquote}}{}
\IfFileExists{microtype.sty}{% use microtype if available
  \usepackage[]{microtype}
  \UseMicrotypeSet[protrusion]{basicmath} % disable protrusion for tt fonts
}{}
\makeatletter
\@ifundefined{KOMAClassName}{% if non-KOMA class
  \IfFileExists{parskip.sty}{%
    \usepackage{parskip}
  }{% else
    \setlength{\parindent}{0pt}
    \setlength{\parskip}{6pt plus 2pt minus 1pt}}
}{% if KOMA class
  \KOMAoptions{parskip=half}}
\makeatother
\usepackage{xcolor}
\newif\ifbibliography
\usepackage{longtable,booktabs,array}
\usepackage{calc} % for calculating minipage widths
\usepackage{caption}
% Make caption package work with longtable
\makeatletter
\def\fnum@table{\tablename~\thetable}
\makeatother
\setlength{\emergencystretch}{3em} % prevent overfull lines
\providecommand{\tightlist}{%
  \setlength{\itemsep}{0pt}\setlength{\parskip}{0pt}}
\setcounter{secnumdepth}{-\maxdimen} % remove section numbering
\def\begincols{\begin{columns}}
\def\begincol{\begin{column}}
\def\endcol{\end{column}}
\def\endcols{\end{columns}}
\ifLuaTeX
  \usepackage{selnolig}  % disable illegal ligatures
\fi
\IfFileExists{bookmark.sty}{\usepackage{bookmark}}{\usepackage{hyperref}}
\IfFileExists{xurl.sty}{\usepackage{xurl}}{} % add URL line breaks if available
\urlstyle{same} % disable monospaced font for URLs
\hypersetup{
  pdftitle={Intervalos de Confianza},
  pdfauthor={Sergio Nava},
  hidelinks,
  pdfcreator={LaTeX via pandoc}}

\title{Intervalos de Confianza}
\subtitle{Estimación puntual y por intervalo}
\author{Sergio Nava}
\date{18/4/2023}

\begin{document}
\frame{\titlepage}

\begin{frame}[allowframebreaks]
  \tableofcontents[hideallsubsections]
\end{frame}
\begin{frame}{Estimadores Puntuales}
\protect\hypertarget{estimadores-puntuales}{}
El objetivo de la estadística inferencial consiste en hacer inferencias
(estimaciones) acerca de los parámetros de una población teniendo en
cuenta la información contenida en la muestra.

La inferencia estadística comprende una serie de técnicas de uso
imprescindible para tomar decisiones con respecto a la cuestión
planteada por el investigador al comienzo de su tarea de análisis de
datos.

Es necesario destacar que, las decisiones que debe tomar el
investigador, ante la situación de incertidumbre, así como, inferir
sobre casos particulares a la generalidad, dichas decisiones deben de
estar basadas en razonamientos que garanticen probabilidades pequeñas de
equivocarse.
\end{frame}

\begin{frame}{Estimación por intervalos}
\protect\hypertarget{estimaciuxf3n-por-intervalos}{}
La estimación por intervalos es un procedimiento mediante el cual se
puede afirmar, con una determinada confianza, que el intervalo \((a,b)\)
encierra el verdadero valor del parámetro. Para realizar una estimación
por intervalos se hace la siguiente afirmación:
\[P(a\le \theta \le b)=1-\alpha\] El intervalo \((a,b)\) se llama
intervalo de confianza, \(b - a\) es una medida de la amplitud de dicho
intervalo y \(1-\alpha\) es una medida de la confianza con la que
contamos para efectuar la estimación.

Nota: \(\alpha\) por sí sola se denomina nivel de significancia.
\end{frame}

\begin{frame}{Intervalos de confianza para medias con varianza conocida}
\protect\hypertarget{intervalos-de-confianza-para-medias-con-varianza-conocida}{}
Hemos explicado que \(\bar{x}\) es el mejor estimador puntual de
\(\mu\), pero nos sorprendería realmente que la media muestral fuera
exactamente igual a \(\mu\). Resultaría más comprensible pensar que el
valor aportado por un estimador se ubica en las cercanías del parámetro.

Esta situación sugiere que puede ser más apropiado efectuar un intervalo
alrededor de \(\bar{x}\) y establecer una cierta confianza de que
\(\mu\) esté comprendido en dicho intervalo.
\end{frame}

\begin{frame}{}
\protect\hypertarget{section}{}
\textbf{Importante}

\begin{itemize}
\tightlist
\item
  En la estimación por intervalos de confianza de un parámetro
  poblacional siempre hablaremos de la confianza de que el intervalo
  contenga al parámetro.
\item
  El parámetro es una cantidad desconocida, pero fija.
\end{itemize}

Ya sabemos que el mejor estimador puntual de \(\mu\) es \(\bar{x}\), la
media muestral y, en consecuencia, lo utilizaremos para la construcción
del Intervalo de confianza. Basándonos en el teorema central del límite
podemos establecer:
\[\bar{X}\sim N \left( \mu,\frac{\sigma^2}{n}\right)\]
\end{frame}

\begin{frame}{}
\protect\hypertarget{section-1}{}
Para poder utilizar la tabla de probabilidades normales debemos
estandarizar esta variable aleatoria.

\[Z = \frac{\bar{X}-\mu}{\frac{\sigma}{\sqrt{n}}} \sim N(0,1)\]

Siendo \(Z\) una variable normal estandarizada, se deberán buscar dos
valores \(z_1\) y \(z_2\) tales que \[ P(z_1 \le Z \le z_2)=1-\alpha \]
o lo que es lo mismo
\[ P\left(z_1 \le \frac{\bar{X}-\mu}{\frac{\sigma}{\sqrt{n}}} \le z_2\right)=1-\alpha \]
\end{frame}

\begin{frame}{}
\protect\hypertarget{section-2}{}
Donde \(z_1 = Z_{1-\frac{\alpha}{2}}\) y \(z_2 = Z_{\frac{\alpha}{2}}\).
Por ejemplo para \(\alpha = 0.05\)

\begin{center}\includegraphics{PresentacionIC_files/figure-beamer/unnamed-chunk-1-1} \end{center}

\emph{Importante} El coeficiente de confianza es la probabilidad
(confianza) de que un Intervalo contenga al parámetro estimado.
\end{frame}

\begin{frame}{}
\protect\hypertarget{section-3}{}
El coeficiente de confianza es un valor fijado por el investigador antes
de comenzar la estimación. Así, si decide trabajar con una confianza del
\(95\%\) para efectuar la estimación, el razonamiento será el siguiente:

\begin{itemize}
\tightlist
\item
  Sobre \(100\) muestras aleatorias de un cierto tamaño \(n\) de una
  población, si en cada una se calcula la medía muestral \(\bar{x}\) y,
  a partir de ellas, se construyen sus correspondientes intervalos de
  confianza para el parámetro que se desea estimar, \(95\) contendrán al
  verdadero valor del parámetro poblacional, mientras que \(5\) no lo
  abarcarán.
\end{itemize}
\end{frame}

\begin{frame}{}
\protect\hypertarget{section-4}{}
\begin{center}\includegraphics{PresentacionIC_files/figure-beamer/unnamed-chunk-2-1} \end{center}

Cada punto segmento pequeño vertical representa la media muestral y cada
línea que pasa a través de estos estos segmentos es la amplitud del
intervalo correspondiente. En este caso vemos que, de 50 intervalos
obtenidos a partir de las 50 muestras aleatorias 5 de tales intervalos
(10\%) no contienen al verdadero valor del parámetro \(\mu\) y 45 (90\%)
sí lo contienen.
\end{frame}

\begin{frame}{}
\protect\hypertarget{section-5}{}
Entonces para \(\alpha = 0.05\) podemos escribir
\[ P\left(-1.96 \le \frac{\bar{X}-\mu}{\frac{\sigma}{\sqrt{n}}} \le 1.96 \right)=0.95 \]

Ahora bien, como estamos tratando de estimar al parámetro \(\mu\) lo
razonable sería despejar convenientemente de modo que quede en el centro
del intervalo el parámetros \(\mu\).

\[ P\left(-1.96 \times \frac{\sigma}{\sqrt{n}} \le \bar{X}-\mu \le 1.96 \times \frac{\sigma}{\sqrt{n}} \right)=0.95 \]
\[ P\left(-1.96 \times \frac{\sigma}{\sqrt{n}} -\bar{X} \le -\mu \le 1.96 \times \frac{\sigma}{\sqrt{n}} - \bar{X} \right)=0.95 \]
\end{frame}

\begin{frame}{}
\protect\hypertarget{section-6}{}
Si multiplicamos por \((-1)\) toda la desigualdad, cambia el sentido de
la desigualdad:
\[ P\left(1.96 \times \frac{\sigma}{\sqrt{n}} +\bar{X} \ge \mu \ge -1.96 \times \frac{\sigma}{\sqrt{n}} + \bar{X} \right)=0.95 \]
es decir
\[ P\left(\bar{X} +1.96 \times \frac{\sigma}{\sqrt{n}}  \ge \mu \ge \bar{X} -1.96 \times \frac{\sigma}{\sqrt{n}} \right)=0.95 \]
reacomodando convenientemente tenemos que
\[ P\left(\bar{X} - 1.96 \times \frac{\sigma}{\sqrt{n}}  \le \mu \le \bar{X} +1.96 \times \frac{\sigma}{\sqrt{n}} \right)=0.95 \]
Este es el intervalo de confianza para el parámetro \(\mu\) cuando
trabajamos con una confianza del \(95\%\).
\end{frame}

\begin{frame}{}
\protect\hypertarget{section-7}{}
\begin{itemize}
\tightlist
\item
  En situaciones reales de investigación, la población generalmente es
  grande y \(\sigma\) es un parámetro desconocido. Para solucionar este
  problema, \(\sigma\) también debe ser estimado. Su estimador lógico
  será \(s\), la desviación estándar de la muestra.
\item
  Si el tamaño de la muestra es suficientemente grande (muestras de
  tamaño mayor o igual a 30), no hay problemas en seguir utilizando la
  distribución de probabilidad Normal para medir la confianza de la
  estimación.
\item
  En cambio, si la muestra es chica y no se puede aumentar su tamaño,
  por razones de costo o de tiempo u otras, para calcular la confianza
  de la estimación utilizaremos otra distribución de probabilidad: la
  distribución correspondiente a la variable \emph{t de Student}, como
  veremos más adelante.
\end{itemize}
\end{frame}

\begin{frame}{}
\protect\hypertarget{section-8}{}
Supongamos, por ejemplo, que se quiere conocer el aumento promedio de
peso de niños pertenecientes a un estrato social muy bajo cuando se los
alimenta con una dieta fortificante.

La población objetivo fue especificada como los niños de dicho estrato
social en el estado de Zacatecas.

Sería casi imposible alimentar con la dieta a todos los niños del
estado. La única solución a este problema será recurrir a una muestra de
\(n = 40\) niños seleccionados aleatoriamente de la población en
cuestión y alimentarlos con la dieta en estudio. Al cabo de un cierto
tiempo se registrarán los aumentos de peso de cada niño y se calculará
el promedio en la muestra.

Supongamos que resultó \(\bar{x}= 4.300\) kg.
\end{frame}

\begin{frame}{}
\protect\hypertarget{section-9}{}
Utilizaremos esta estimación puntual para efectuar una estimación por
intervalos del parámetro \(\mu\): aumento de peso promedio de niños
pertenecientes al estrato social muy bajo del estado de Zacatecas.

Supongamos que, por estudios previos se conoce que el valor que toma el
parámetro varianza poblacional es \(\sigma^2=4\) \(kg^2\). Podemos
utilizar la distribución de probabilidad normal para efectuar la
estimación. Con una confianza de \(1-\alpha=0.95\), el intervalo
resultante será:
\[ P\left(4.3 - 1.96 \times \frac{2}{\sqrt{40}}  \le \mu \le 4.3 +1.96 \times \frac{2}{\sqrt{40}} \right)=0.95 \]
\[P(3.69 \le \mu \le 4.69) = 0.95\] Y nos da una amplitud del intervalo
\(4.96-3.68=1.24\)
\end{frame}

\begin{frame}{}
\protect\hypertarget{section-10}{}
Ahora bien, si en lugar de tomar una muestra de \(40\) niños hubiésemos
tomado una de \(n = 100\) niños, manteniendo todos los demás valores
constantes, el intervalo será:
\[ P\left(4.3 - 1.96 \times \frac{2}{\sqrt{100}}  \le \mu \le 4.3 +1.96 \times \frac{2}{\sqrt{100}} \right)=0.95 \]
\[P(3.91 \le \mu \le 4.69) = 0.95\] Y nos da una amplitud de
\(4.69 - 3.91=0.78\).

Vemos que al aumentar el tamaño de la muestra disminuyó la amplitud del
intervalo de confianza, logrando así una mayor precisión en la
estimación.
\end{frame}

\begin{frame}{}
\protect\hypertarget{section-11}{}
Pero, ¿Qué pasaría ahora si con los mismos datos la varianza poblacional
\(\sigma^2\) fuese 9 en lugar de 4?

En este caso estamos pensando que la variable aumento de peso en los
niños tiene una variabilidad mayor de niño a niño.

\[ P\left(4.3 - 1.96 \times \frac{3}{\sqrt{40}}  \le \mu \le 4.3 +1.96 \times \frac{3}{\sqrt{40}} \right)=0.95 \]
\[P(3.37 \le \mu \le 5.23) = 0.95\] Lo que nos da una amplitud de
intervalo de \(5.23-3.37=1.86\).

Si comparamos este intervalo con el primero que hemos construido, vemos
que ahora se han ampliado los límites provocando una merma en la
precisión de la estimación.
\end{frame}

\begin{frame}{}
\protect\hypertarget{section-12}{}
\textbf{Y ¿Qué podemos decir de la confianza de la estimación?}

Cuando se efectúa una estimación por intervalos, a medida que se
incrementa el nivel de confianza se incrementa también la amplitud del
intervalo y, en consecuencia, disminuye la precisión del mismo.

Así, por ejemplo, si decidimos aumentar el \textbf{nivel de confianza}
trabajando con \(1-0.01=\)\textbf{0.99}, el intervalo sería:

\[ P\left(4.3 - 2.576 \times \frac{2}{\sqrt{40}}  \le \mu \le 4.3 +2.576 \times \frac{2}{\sqrt{40}} \right)=0.99 \]
\[P(3.49 \le \mu \le 5.11) = 0.99\] Lo que nos da una amplitud de
intervalo de \(5.11-3.49=\) \(1.62\). Luego, construyendo el intervalo
con un \(99\%\) de confianza hemos aumentado nuestra seguridad en la
estimación, pero a costa de una precisión menor.
\end{frame}

\begin{frame}{}
\protect\hypertarget{section-13}{}
Pero, para evitar muchos cálculos y aprovechando la simetría de la
distribución normal se expone la ecuación siguiente para el cálculo del
intervalo de confianza de la media \((n\ge30)\):
\[ P\left(\bar{X} - Z_\frac{\alpha}{2} \times \frac{\sigma}{\sqrt{n}}  \le \mu \le \bar{X} + Z_\frac{\alpha}{2} \times \frac{\sigma}{\sqrt{n}} \right)=1-\alpha \]
Algunos autores expresan el intervalo como
\(\bar{X} \pm Z_\frac{\alpha}{2} \times \frac{\sigma}{\sqrt{n}}\)

\begin{longtable}[]{@{}rrr@{}}
\toprule()
Alfa & Coeficiente de Confianza & Z \\
\midrule()
\endhead
0.10 & 0.90 & 1.645 \\
0.05 & 0.95 & 1.960 \\
0.01 & 0.99 & 2.576 \\
\bottomrule()
\end{longtable}
\end{frame}

\begin{frame}{Intervalo de la media con varianza desconocida}
\protect\hypertarget{intervalo-de-la-media-con-varianza-desconocida}{}
un caso muy común es cuando el investigador está obligado a trabajar con
muestras chicas y pretende estimar la media poblacional con \(\sigma^2\)
desconocida.

En estos casos, dado que no se conoce la \(\sigma^2\), para estimar por
intervalos al parámetro \(\mu\) debemos recurrir a un nuevo estadístico
distribuido como una variable \emph{``t'' de Student}. Este estadístico
surge de estandarizar la variable media muestral, pero tomando a \(S^2\)
como estimador puntual de la varianza poblacional.
\[t=\frac{\bar{X}-\mu}{\frac{S}{\sqrt{n}}}\]
\end{frame}

\begin{frame}{}
\protect\hypertarget{section-14}{}
Esta variable ya no se distribuye normalmente cuando el tamaño de la
muestra es chico, en la práctica \(n<30\). Entonces
\[\frac{\bar{X}-\mu}{\frac{S}{\sqrt{n}}} \sim t_{n-1}\] Donde \(n-1\),
es el denominador con el que se calculó la varianza muestral \(S^2\).
Este valor corresponde a los grados de libertad de la variable ``t'' de
Student.

Similarmente al caso de la normal obtenemos:
\[ P\left(\bar{X} - t_{\frac{\alpha}{2},n-1} \times \frac{S}{\sqrt{n}}  \le \mu \le \bar{X} + t_{\frac{\alpha}{2},n-1} \times \frac{S}{\sqrt{n}} \right)=1-\alpha \]
Algunos autores expresan el intervalo como
\(\bar{X} \pm t_{\frac{\alpha}{2},n-1} \times \frac{\sigma}{\sqrt{n}}\)
\end{frame}

\begin{frame}{}
\protect\hypertarget{section-15}{}
Supongamos que un investigador desea estimar el rendimiento promedio de
grasa que contiene la leche producida por vacas de cierta raza en un
período de tiempo determinado.

Para ello se extrae una muestra de 10 vacas lecheras, obteniendo:

\(n=10\) , \(\bar{X}=36.4 grs\), \(s^2=264.04 grs^2\) y \(s=16.25 grs\)

Por lo tanto \(s^2 /n = 264.04/10=26.404\) y \(s/\sqrt{n}=5.138 grs\)

Los únicos datos que posee el investigador para llevar a cabo su
investigación son la media y la varianza muestral.

Como la muestra es chica y no conocemos la varianza poblacional
\(\sigma^2\), el estadístico para confeccionar el correspondiente
intervalo de confianza de \(\mu\), tendrá distribución \emph{t de
Student} con \(n-1\) grados de libertad.
\end{frame}

\begin{frame}{}
\protect\hypertarget{section-16}{}
Para un nivel de confianza del \(95\%\) y buscando convenientemente en
la tabla de probabilidades de la distribución \emph{t de Student},
obtenemos:\[t_{\frac{0.05}{2},10-1}=t_{0.025,9}=2.62\]

El intervalo de confianza será
\[ P\left(\bar{X} - t_{\frac{\alpha}{2},n-1} \times \frac{S}{\sqrt{n}}  \le \mu \le \bar{X} + t_{\frac{\alpha}{2},n-1} \times \frac{S}{\sqrt{n}} \right)=0.95 \]
\[ P\left(36.4 - 2.62 \times 5.138  \le \mu \le 36.4 + 2.62 \times 5.138 \right)=0.95 \]
\[ P\left(24.78  \le \mu \le 48.02 \right)=0.95 \] El promedio de grasa
en la leche de dichos animales se estima entre un \(24.78\) grs y
\(48.02\) grs con un nivel de confianza del \(95\%\).
\end{frame}

\begin{frame}{Intervalos de confianza para proporciones}
\protect\hypertarget{intervalos-de-confianza-para-proporciones}{}
Lo visto en la sección anterior se puede usar con el fin de determinar
intervalos de confianza para la media de cualquier población se la que
se haya extraido una muestra grande. Cuando la población tiene una
distribución de Bernoulli, esta expresión toma una forma especial.

Suponga que se tiene un proceso de fabricación y este tiene
especificaciones. Se prueba una muestra de 144 productos y 120
satisfacen la especificación. Sea \(p\) la proporción de artículos en la
población que satisfacen dicha especificación. Se desea encontrar un
intervalo de confianza de \(95\%\) para \(p\).
\end{frame}

\begin{frame}{}
\protect\hypertarget{section-17}{}
Se empieza construyendo un estimador de \(p\). Sea \(X\) el número de
artículos en la muestra que satisface la especificación. Entonces
\(X \sim Bin(n, p)\), donde \(n = 144\) es el tamaño muestral. El
estimador de \(p\) es \(\hat{p}= X/n\). En este ejemplo, \(X = 120\),
por lo que \(\hat{p}= 120/144 = 0.833\). La incertidumbre, o desviación
estándar de \(\hat{p}\), es \(\sigma_{\hat{p}}=\sqrt{p(1-p)}\). Puesto
que el tamaño muestral es grande, se tiene por el teorema del límite
central que \[\hat{p} \sim N\left(p,\frac{p(1-p)}{n}  \right) \]

Con un razonamiento similar a los casos anteriores se puede obtener
\[ P\left( \hat{p} -Z_{\frac{\alpha}{2} } \sqrt{\frac{p(1-p)}{n} } \le p \le \hat{p} + Z_{\frac{\alpha}{2} } \sqrt{\frac{p(1-p)}{n} }\right) =1- \alpha\]
\end{frame}

\begin{frame}{}
\protect\hypertarget{section-18}{}
A primera vista la expresión anterior parece un intervalo de confianza
con un nivel \(1-\alpha\) para \(p\). Sin embargo los límites contienen
una \(p\) desconocida, y por eso no se puede calcular. El punto de vista
tradicional es sustituir esa \(p\) por \(\hat{p}\). De tal manera que
quedaría así:
\[ P\left( \hat{p} -Z_{\frac{\alpha}{2} } \sqrt{\frac{ \hat{p}(1- \hat{p})}{n} } \le p \le \hat{p} + Z_{\frac{\alpha}{2} } \sqrt{\frac{ \hat{p}(1- \hat{p})}{n} }\right) =1- \alpha\]

Algunos autores expresan el intervalo como
\(\hat{p} \pm Z_\frac{\alpha}{2} \times \sqrt{\frac{\hat{p}(1-\hat{p})}{n}}\)

Para el ejemplo obtenemos
\(.833 \pm 1.96\times\sqrt{.833\times(1-.833)/144}\) es decir
\(0.833 \pm 0.0609\) y por lo tanto el intervalo va de \(0.7721\) a
\(0.8939\) con una confianza del \(95\%\).
\end{frame}

\begin{frame}{Intervalos de confianza para la varianza}
\protect\hypertarget{intervalos-de-confianza-para-la-varianza}{}
Para utilizar a \(S^2\) como estimador de \(\sigma^2\) necesitamos
conocer su distribución de probabilidad. De esta manera podremos
establecer un cierto coeficiente de confianza de la estimación.

No existe una distribución conocida para \(S^2\) pero sí para cierta
transformación del mismo. Si la muestra proviene de una población en la
cual la variable en estudio se distribuye normalmente, tenemos
\[(n-1)\frac{S^2}{\sigma^2} \sim \chi_{n-1}^2 \] donde \(\chi_{n-1}^2\)
es la distribución Ji-cuadrada co \(n-1\) grados de libertad.
\end{frame}

\begin{frame}{}
\protect\hypertarget{section-19}{}
Una vez que contamos con esta información, podemos establecer un
intervalo de confianza para estimar \(\sigma^2\), de la siguiente
manera:
\[ P\left( \chi_{1-\frac{\alpha}{2},n-1}^2 \le (n-1)\frac{S^2}{\sigma^2} \le \chi_{\frac{\alpha}{2},n-1}^2 \right)  = 1-\alpha \]

Despejando tenemos
\[ P\left( \frac{S^2 \times (n-1)}{\chi_{\frac{\alpha}{2},n-1}^2} \le \sigma^2 \le\frac{S^2 \times (n-1)}{\chi_{1-\frac{\alpha}{2},n-1}^2} \right)  = 1-\alpha \]
\end{frame}

\begin{frame}{}
\protect\hypertarget{section-20}{}
\textbf{Ejemplo}

El Departamento de Control de Calidad de una empresa decide estudiar la
homogeneidad de los productos comprados a un proveedor de materiales
plásticos. Para ello toma 6 piezas de la producción de un envío y las
somete a pruebas de resistencia. Las siguientes observaciones
representan las cargas de rotura en unidades de 1,000 libras por pulgada
cuadrada: 15.3, 18.7, 22.3, 17.6, 19.1 y 14.8

El Departamento de Calidad ha determinado que interesa conocer la
variabilidad de las cargas de rotura, pues la uniformidad de este insumo
es fundamental en el proceso de producción de la fábrica.

Como primera medida y habiendo comprobado que la variable cargas de
rotura tiene distribución normal. Se procede a construir un intervalo de
confianza del \(90\%\):
\end{frame}

\begin{frame}{}
\protect\hypertarget{section-21}{}
\(S^2=7.575\), \(n-1=5\), \(\alpha=0.1\) por lo tanto
\(\chi_{.05,5}=11.07\) y \(\chi_{.95,5}=1.15\), y el intervalo de
confianza sería
\[ P\left( \frac{7.575 \times (5)}{\chi_{\frac{0.10}{2},5}^2} \le \sigma^2 \le\frac{7.575 \times (5)}{\chi_{1-\frac{0.10}{2},5}^2} \right)  = 0.90 \]
\[ P\left( 3.42 \le \sigma^2 \le 32.93 \right)  = 0.90 \]

Es decir el intevalo es de confianza de la varianza a un nivel de
\(90\%\) es \((3.42,32.93)\).
\end{frame}

\begin{frame}{Obtención de Intervalos de confianza con R}
\protect\hypertarget{obtenciuxf3n-de-intervalos-de-confianza-con-r}{}
\begin{longtable}[]{@{}
  >{\raggedright\arraybackslash}p{(\columnwidth - 6\tabcolsep) * \real{0.1275}}
  >{\raggedright\arraybackslash}p{(\columnwidth - 6\tabcolsep) * \real{0.1176}}
  >{\raggedright\arraybackslash}p{(\columnwidth - 6\tabcolsep) * \real{0.2843}}
  >{\raggedright\arraybackslash}p{(\columnwidth - 6\tabcolsep) * \real{0.4706}}@{}}
\toprule()
\begin{minipage}[b]{\linewidth}\raggedright
Una muestra
\end{minipage} & \begin{minipage}[b]{\linewidth}\raggedright
Parámetro
\end{minipage} & \begin{minipage}[b]{\linewidth}\raggedright
Función
\end{minipage} & \begin{minipage}[b]{\linewidth}\raggedright
Ejemplo
\end{minipage} \\
\midrule()
\endhead
Media & \(\mu\) & t.test & t.test(x=altura,
conf.level=0.90)\$conf.int \\
Proporción & \(p\) & prop.test & prop.test(x=275, n=500,
conf.level=0.90)\$conf.int \\
Varianza & \(\sigma^2\) & stests::var.test() & var.test(x=altura,
conf.level=0.98)\$conf.int \\
& & & \\
\bottomrule()
\end{longtable}
\end{frame}

\begin{frame}{}
\protect\hypertarget{section-22}{}
\begin{longtable}[]{@{}
  >{\raggedright\arraybackslash}p{(\columnwidth - 6\tabcolsep) * \real{0.2011}}
  >{\raggedright\arraybackslash}p{(\columnwidth - 6\tabcolsep) * \real{0.0862}}
  >{\raggedright\arraybackslash}p{(\columnwidth - 6\tabcolsep) * \real{0.1092}}
  >{\raggedright\arraybackslash}p{(\columnwidth - 6\tabcolsep) * \real{0.6034}}@{}}
\toprule()
\begin{minipage}[b]{\linewidth}\raggedright
Diferencia de
\end{minipage} & \begin{minipage}[b]{\linewidth}\raggedright
Parámetro
\end{minipage} & \begin{minipage}[b]{\linewidth}\raggedright
Función
\end{minipage} & \begin{minipage}[b]{\linewidth}\raggedright
Ejemplo
\end{minipage} \\
\midrule()
\endhead
Medias de muestras independientes & \(\mu_1-\mu_2\) & t.test &
t.test(x=hombres\$altura, y=mujeres\$altura,,paired=FALSE,
var.equal=FALSE,conf.level = 0.95)\$conf.int \\
Medias de muestras pareadas & \(\mu_1-\mu_2\) & t.test & t.test(x=Antes,
y=Despues, paired=TRUE, conf.level=0.95)\$conf.int \\
Proporciones & \(p_1-p_2\) & prop.test & prop.test(x=c(75, 80),
n=c(1500, 2000), conf.level=0.90)\$conf.int \\
Varianza & \(\sigma^2\) & stests::var.test() &
var.test(x=hombres\$altura,
y=mujeres\$altura,,conf.level=0.95)\$conf.int \\
\bottomrule()
\end{longtable}

\href{https://fhernanb.github.io/Manual-de-R/ic.html}{Intervalos de
confianza en R}
\end{frame}

\end{document}
